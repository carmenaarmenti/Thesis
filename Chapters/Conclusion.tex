\chapter{Conclusion}
In this work we wanted to test the curriculum learning approach on code related tasks.
In order to test this new technique we decided to take as reference two important papers and to compare 
our findings to the results achieved only by using deep learning models.
As explained in Chapter \ref{chapter:MAO} there are well-known advantages that curriculum learning
brougth to the current state of the art. However, it is also true that advantages and benefits depend highly on the data 
the model takes in input and consequently the results with curriculum applied might not always be better than 
those obtained on the plain model. 
In fact, we could notice that if on bug-fixing task the model with curriculum learning 
took better results not only in terms of performance but also of accuracy, 
in the case of the second task, i.e. code summarization, the results achieved in \cite{Leclair2020}
are definetly better.
Nevertheless, it must be said that we focused our attention on the technique used to train, but in order to achieve better 
results other changes at the baseline model migh be done. To illustrate, if for the first task we used the very same model 
to train both the baseline and the curriculum learning approach, in the second case the model used as for the training was not the 
one used by the authors in \cite{Leclair2020}. The negative results achieved might indicate differen scenarios. One possible reason might be that 
the approach used is not suitable for code-summarization-like tasks, since it is the case of a NLP objective to learn for the model.
A more robust technique might bring positive results.
Moreover, those negative results might not be that negative. We proposed an approach that worked very well on a task that was originally tested on the same 
training model. It was not the case for the second one. Reproducing the experiment on the same model used by LeClair \cite{Leclair2020} may possibly bring different results.
If that was the case, the curriculum learning would be an approach suitable for any deep learning model.
To sum up, by experimenting two tasks and collecting one positve and one negative result we can acknowledge that curriculum learning 
outcomes highly depend on the training data and the approach must be settled differently for each task.
