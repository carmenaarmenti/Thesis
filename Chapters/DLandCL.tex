\chapter{Deep Learning and Curriculum Learning}
Inspired by human learning, curriculum learning is an algorithm that emphasizes
the order of training instances in a computational learning setup.
As a feature of human learning, curriculum, or even better learning in a meaningful way,
has been transferred to machine learning, thus creating the subdiscipline named
\textit{curriculum learning} \cite{Wang2020}.\\
Essentially, human education is organized as curricula, by starting small indeed, and gradually presenting more complex
concepts. The paramount hypothesis is that simpler instances should be learned
during the first steps as building blocks to then learn more complex ones. Several experiments on sentiment 
analysis task and tasks similar to sequence prediction tasks in NLP carried on by Cirick \textit{et al} \cite{Cirik2016VisualizingAU} prove that
curriculum learning has positive effects on LSTM's internal states, by biasing the model through building constructive representations. 
Specifically, the internal representation at the previous timestep is used as building block for the next one, thus
contributing at the final prediction.


\section{State of the art}
\section{Curriculum Learning related works}
\section{Curriculum Learning application in Deep Learning tasks}



%%%%%%%%%%%%%%%%%%%%
% NOTES
% As mentioned previously, this is the first work studying LSTM networks
% on software engineering tasks with curriculum learning to our knowledge.

%%%%%%%%%%%%%%%%%%%%